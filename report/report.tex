\documentclass[12pt,a4paper]{article}
\usepackage{amsmath}
\usepackage{amssymb}
\usepackage{amsfonts}
\usepackage{graphicx}
\usepackage{fancyhdr}
\pagestyle{fancy}
\fancyhf{}
\rhead{Ling Fei Zhang, Brandon Ma}
\lhead{Artificial Intelligence}
\cfoot{\thepage}
\author{Ling Fei Zhang, 260985358\\
Brandon Ma, 260983550}
\begin{document}
\title{AI Final Project Report Winter 2022}    
\maketitle


\section{Motivation}
\subsection{Approach}
\paragraph{March 25th 2022}
Today, we played a few games against each other. We believe that in order to design a good AI agent, 
we must first understand the fundamentals of the game as well as to develop a few basic strategies. 
Our first instinct is to avoid 
corners as we can easily get boxed in. This means that generally speaking, we would 
like to move towards the center. 


\section{Theoretical Approach}
\subsection{Basic Strategies}
First, we need to implement some basic strategies. We implemented a Depth First Search in order to enumerate 
all the possible moves given a specific position, a chess board and a maximum number of steps. Below we 
have some other basic strategies:
\begin{enumerate}
    \item At every turn, our agent checks if it can win in 1 move. 
    \item Trying to box in the opponent as much as possible.
    \item Use our agent as a wall to limit the opponent's movement.
    \item Make an heursitic function that is balanced between centralization, not getting trapped and cornerning the oppponent. 
\end{enumerate}
\subsection{More Complexe Functions}
We also implemented some other functions that are more complexe, such as minimax and alpha-beta pruning. 
The goal is to use a heuristic function to give a score to each move. This implies that the agent can 
decide which move has better winning chances relative to others. We also have functions to count each player's 
potential territory. This is particularly useful in endgames, where 1 or 2 walls in the world ends the game. 

\section{Advantages and Disadvantages}
\subsection{Advantages}
In our algorithm, on top of minimax, we have implemented a mating search. This means that up to a certain depth, 
our algorithm can find a forced win by choosing a sequence of moves (very much like Chess). This function is used 
in conjuction with the minimax and alpha-beta pruning. 
\subsection{Disadvantages}
One of the disadvantages of our approach using minimax is that this algorithm assumes that the opponent plays 
the best possible move when it's their turn. This implies that if the opponent is not optimized (i.e. versus a 
random agent or event another student's agent), and doesn't play the best move, then we could've potentially 
obtained a better score by choosing another move. 
\subsection{Expected Failure}
Some expected failures. 
\subsection{Weaknesses}
Some weaknesses. 

\section{Other Approaches}
We've tried other approaches at the beginning such as using a Breadth First Search in order to generate the 
tree of possible moves. We realized that at each iteration of a Breadth First Search, 
we would also have to keep track of 
all the nodes up to the depth that we are searching for. On the other hand, an Iterative Deepening Search will not 
have to keep all of them in memory since either some nodes are already visited or they are nodes that we will 
visit later. Not to mention, Iterative Deepening Search also does better in our situation since we have a time 
constraint. In the event that we reach the time limit, the algorithm will still 
have some sense of what a good score is even without 
a full traversal of the desired depth. 

\section{Further Improvements}
We believe that one improvement to our agent can be improving its heuristic with some added features to consider 
such as: 
\begin{enumerate}
    \item some extra feature
\end{enumerate}












% \paragraph{} We also discussed about strategies to inplement our code. First, we need a 
% heuristic function for our agent to evaluate the current position. Given the maximum steps 
% allowed and the heuristic function, the agent will compute an evaluation for each of the 
% reachable squares. The agent will then pick the best square and make its move. Our heuristic 
% function will be based on the following ideas:
% \begin{enumerate}
%     \item At every turn, agent checks if we can win in 1 (or 2 if allowed) move(s). 
%     \item In case our heuristic function of a tie between two squares, our agent will prefer to move towards the center.
%     \item check how many walls are around our agent with a radius of 1 step. 
%     If our agent has many walls next to it, it should try and escape.
%     \item Use minimax and alpha-beta pruning near end-game so that we can search at a deeper depth. 
%     \item Use BFS when evaluating minimax so that we can maximize the depth computed given a time limit. 
%     \item Trying to box in the opponent as much as possible.
%     \item Use our agent as a wall to limit the opponent's movement.
%     \item If the opponent is surround by two walls, try to box the opponent towards the edge of the chess board
%     \item A heuristic based on the number of squares you control (i.e. the squares you can 
%     get to with your moves). The squares that both players control don't count. 
%     \item Opening as player A:
%     \begin{enumerate}
%         \item Move towards the middle, and do NOT put a wall towards the opponent. 
%         The idea is to move towards the center and control the oppoenent's "territory". 
%     \end{enumerate}
%     \item Opening as player B: 
%     \item Middle game: 
%     \item End game: 
%     \begin{enumerate}
%         \item if there exist a wall in the board that we can place to completely divide the two players, then we're 
%         in an end-game.
%         \item (Total number of possible walls in the areas that the players can go to) - constant
%         \item NEED QUANTIFIER TO CHECK HOW MANY SQUARES WE'RE LOSING
%         \item Want to go towards the closing area.
%     \end{enumerate}
%     \item When there are a lot of walls on the board, use minimax with alpha-beta pruning at very highg depths
% \end{enumerate}
% dddddd


% Balance between \\
% 1) centralization\\
% 2) not getting trapped\\
% 3) aggressive (blocking opponent's moves)\\

\end{document}
