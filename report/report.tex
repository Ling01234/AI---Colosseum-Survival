\documentclass[12pt,a4paper]{article}
\usepackage{amsmath}
\usepackage{amssymb}
\usepackage{amsfonts}
\usepackage{graphicx}
\author{Ling Fei Zhang, 260985358\\
Brandon Ma, 260983550}
\begin{document}
\title{AI Final Project Report Winter 2022}    
\maketitle
\section{Motivation}
\subsection{Approach}
\paragraph{March 25th 2022}
Today, we played a few games against each other in order to understand the fundamentals 
of the game as well as to develop a few basic strategies. Our first instinct is to avoid 
corners as we can easily get boxed in if we are near corners. This means that we would 
like to move towards the center. 
\paragraph{} We also discussed about strategies to inplement our code. First, we need a 
heuristic function for our agent to evaluate the current position. Given the maximum steps 
allowed and the heuristic function, the agent will compute an evaluation for each of the 
reachable squares. The agent will then pick the best square and make its move. Our heuristic 
function will be based on the following ideas:
\begin{enumerate}
    \item In case our heuristic function of a tie between two squares, our agent will prefer to move towards the center.
    \item Trying to box in the opponent as much as possible.
    \item Use our agent as a wall to limit the opponent's movement.
    \item If the opponent is surround by two walls, try to box the opponent towards the edge of the chess board
    \item When there are a lot of walls on the board, use minimax with alpha-beta pruning at very highg depths
\end{enumerate}


\end{document}
