\documentclass[12pt,a4paper]{article}
\usepackage{amsmath}
\usepackage{amssymb}
\usepackage{amsfonts}
\usepackage{graphicx}
\author{Ling Fei Zhang, 260985358\\
Brandon Ma, 260983550}
\begin{document}
\title{AI Final Project Report Winter 2022}    
\maketitle
\section{Motivation}
\subsection{Approach}
\paragraph{March 25th 2022}
Today, we played a few games against each other in order to understand the fundamentals 
of the game as well as to develop a few basic strategies. Our first instinct is to avoid 
corners as we can easily get boxed in if we are near corners. This means that we would 
like to move towards the center. 
\paragraph{} We also discussed about strategies to inplement our code. First, we need a 
heuristic function for our agent to evaluate the current position. Given the maximum steps 
allowed and the heuristic function, the agent will compute an evaluation for each of the 
reachable squares. The agent will then pick the best square and make its move. Our heuristic 
function will be based on the following ideas:
\begin{enumerate}
    \item At every turn, agent checks if we can win in 1 (or 2 if allowed) move(s). 
    \item In case our heuristic function of a tie between two squares, our agent will prefer to move towards the center.
    \item check how many walls are around our agent with a radius of 1 step. 
    If our agent has many walls next to it, it should try and escape.
    \item Use minimax and alpha-beta pruning near end-game so that we can search at a deeper depth. 
    \item Use BFS when evaluating minimax so that we can maximize the depth computed given a time limit. 
    \item Trying to box in the opponent as much as possible.
    \item Use our agent as a wall to limit the opponent's movement.
    \item If the opponent is surround by two walls, try to box the opponent towards the edge of the chess board
    \item A heuristic based on the number of squares you control (i.e. the squares you can 
    get to with your moves). The squares that both players control don't count. 
    \item Opening as player A:
    \begin{enumerate}
        \item Move towards the middle, and do NOT put a wall towards the opponent. 
        The idea is to move towards the center and control the oppoenent's "territory". 
    \end{enumerate}
    \item Opening as player B: 
    \item Middle game: 
    \item End game: 
    \begin{enumerate}
        \item if there exist a wall in the board that we can place to completely divide the two players, then we're 
        in an end-game.
        \item (Total number of possible walls in the areas that the players can go to) - constant
        \item NEED QUANTIFIER TO CHECK HOW MANY SQUARES WE'RE LOSING
        \item Want to go towards the closing area.
    \end{enumerate}
\end{enumerate}
dddddd


Balance between \\
1) centralization\\
2) not getting trapped\\
3) aggressive (blocking opponent's moves)\\

\end{document}
